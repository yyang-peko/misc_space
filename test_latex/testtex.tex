\documentclass[12pt]{article}
\usepackage[paper=a4paper]{geometry}
\usepackage[utf8]{inputenc}
\usepackage{pdfpages}

\usepackage{todonotes}

\usepackage{ragged2e,array,url,graphicx,csquotes,hyperref,adjustbox,longtable}
\usepackage[english]{babel}

\usepackage{setspace}

\renewcommand{\topfraction}{.85}
\renewcommand{\bottomfraction}{.7}
\renewcommand{\textfraction}{.15}
\renewcommand{\floatpagefraction}{.66}
\renewcommand{\dbltopfraction}{.66}
\renewcommand{\dblfloatpagefraction}{.66}
\setcounter{topnumber}{9}
\setcounter{bottomnumber}{9}
\setcounter{totalnumber}{20}
\setcounter{dbltopnumber}{9}

\setstretch{2}

\title{Your Paper}
\author{You}
\setlength{\marginparwidth}{2cm}
\begin{document}
\maketitle

\begin{abstract}
Your abstract.
\end{abstract}

\section{Introduction}

Your introduction goes here! Some examples of commonly used commands and features are listed below, to help you get started.

Using LateX is very simple, this template already included a lot of the different packages so you don't have to worry. When making your own latex document it is often easier to start with a template than starting from scratch, whether that is your own created template, or one that is freely available online.

\subsection{Some basics}

Using the \verb=\section= and \verb=\subsection= commands you can define the paragraph and subparagraph names, like '1 Introduction' and '1.1 some basics' above. 
When you keep an empty line in between two blocks of text, the lower one will have an indentation

like this.

\section{Some examples to get started}

\subsection{How to add Lists}

You can make lists with automatic numbering \dots

\begin{enumerate}
\item Like this,
\item and like this.
\end{enumerate}
\dots or bullet points \dots
\begin{itemize}
\item Like this,
\item and like this.
\end{itemize}

\end{document}